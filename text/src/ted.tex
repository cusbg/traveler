%\newcommand{\F}{\mathbb{F}}
%\newcommand{\G}{\mathbb{G}}

\chapter{Tree-edit-distance algoritmus}

Jadro aplikacie lezi v pouziti tree-edit-distance (TED) algoritmu,
vdaka ktoremu dostaneme mapovanie medzi 2 RNA stromami. Mapovanie nam ukaze
spolocne casti oboch RNA stromov. TED algoritmus je obdoba Levenstheinoveho
string-edit-distance algoritmu. Problem u retazcov je specialnym pripadom
TED-u, kedy stromy zdegenerovali na cesty (spojovy zoznam).

\section{Hlavna myslienka TED-u}

Zakladom TED algoritmu je v rekurzivnom vzorci \ref{eq:ted}. Vzdialenost medzi
lesmi F a G, $\delta(F, G)$ je definovana ako minimalny pocet editacnych operacii,
ktore z F urobia G. Pouzivame standardne editacne operacie - delete, insert, update.
Delete, zmazanie vrcholu, znamena pripojit k predkovi vsetkych jeho potomkov so
zachovanim poradia medzi nimi. Insert, vlozenie vrcholu, je opacna operacia k
delete, co znamena, ze vkladame vrchol medzi rodica nejakych jeho, po sebe
nasledujucich potomkov. Update iba zmeni hodnotu vo vrchole stromu.

\begin{figure}[H]
\begin{subequations}
\begin{align}
	\begin{split}
	\delta(\emptyset, \emptyset) &=
		0
		\\
	\delta(F, \emptyset) &=
		\delta(F - r_{F}, \emptyset) + c_{del}(r_{F})
		\\
	\delta(\emptyset, G) &=
		\delta(\emptyset, G - r_{G}) + c_{ins}(r_{G})
	\end{split}
	\\[1ex]
	\delta(F, G) &=
		\begin{cases}
			\delta(F - r_{F}, G) + c_{del}(r_{F}) \\
			\delta(F, G - r_{G}) + c_{ins}(r_{G}) \\
			\!
			\begin{aligned}
				\delta(F -& F_{r_{F}}, G - G_{r_{G}}) + \\
				& \delta(F_{r_{F}} - r_{F}, G_{r_{G}} - r_{G}) + c_{upd}(r_{F}, r_{G})
			\end{aligned}
		\end{cases}
\end{align}
\end{subequations}
\label{eq:ted} \caption{Rekurzivny vzorec pre vypocet tree-edit-distance}
\end{figure}

\citet{RTED}
\\



