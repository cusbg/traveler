
\chapter*{Úvod}
\addcontentsline{toc}{chapter}{Úvod}

Donedávna sa myslelo, že úloha ribonukleovej kyseliny, RNA, je obmedzená
iba na syntézu bielkovín, buď ako nositeľka genetickej informácie (mRNA),
alebo ako prenášač aminokyselín pri ich tvorbe (tRNA).
Avšak existuje mnoho ďalších druhov, od relatívne malých molekúl majúcich
iba desiatky nukleotidov, ktoré ovplyvňujú expresiu génov (miRNA, siRNA, tmRNA, snRNA a ďalšie),
až po veľké molekuly s tisíckami nukleotidov, ktoré sa podieľajú na tvorbe ribozómu (rRNA).

Spolu s objavmi ďalších a ďalších funkcií RNA molekúl rastie záujem o nástroje dovoľujúce
študovať štruktúru týchto molekúl.
Primárna štruktúra je určená poradím nukleotidov v reťazci RNA. Priestorovým usporiadaním
získame terciárnu štruktúru. Poslednou a v tejto práci pre nás najdôležitejšou
bude sekundárna štruktúra. Tú reprezentuje zoznam nukleotidov ktoré sú spojené väzbou.
Spárované nukleotidy sú blízko seba v priestore a tak nám sekundárna relatívne
dobre aproximuje terciárnu štruktúru. Predpovedanie terciárnej štruktúry nieje veľmi
spoľahlivé ani pre kratšie molekuly. Naopak, pre menšie molekuly a ich  sekundárnu
štruktúru existujú spoľahlivejšie metódy, ktorých prehľad a porovnanie nájdeme napríklad
v \citenum{SEC_STR_PREDICT_TOOLS}.
Príbuzné štruktúry nám vedia poslúžiť k predpovedaniu konzervovaných častí, dokonca
aj veľkých rRNA molekúl \citenum{SEC_STR_PREDICTION}.

Vizualizácia sekundárnej štruktúry RNA sa dá previesť na kreslenie grafu,
ktorého vrcholy tvoria nukleotidy a hrany reprezentujú páry medzi nimi.
Kreslenie grafov je značne preskúmanou témou, keďže nachádza uplatnenie vo veľa
doménach, ako napríklad analýza sociálnych sietí \citenum{SOCIAL_NETWORK_ANALYSIS}
alebo vo všeobecnej analýze dát \citenum{GRAPH_DRAWING}.

Cieľom vizualizácie sekundárnej štruktúry RNA je zachytit párovanie nukleotidov
v molekule a ideálne všetky ďalšie motívy, ktoré sa v štruktúre vyskytujú,
ako napríklad hairpin, bulge, interior a multi-branch loop.
Existujúce nástroje na vizualizáciu zväčša volia medzi tromi tipmi reprezentácie
štruktúry \citenum{JVIZ}: spojnicový graf, kruhový graf alebo štandardná štruktúra.
Aj keď spojnicový a kruhový graf podporujú vizualizáciu párovania báz, motívy
sa v nej dajú nájsť len veľmi ťažko, ak vôbec.
Preto nám na hľadanie motívov v RNA ostala štandardná reprezentácia štruktúry RNA.
Bolo vymyslených množstvo riešení -
RNAplot z balíka ViennaRNA \citenum{VIENNA_RNA}, VARNA \citenum{VARNA},
RnaViz \citenum{RNA_VIZ}, jViz.RNA \citenum{JVIZ}, mfold \citenum{MFOLD},
XRna \citenum{TODOcitacia}, PseudoViewer \citenum{PSEUDOVIEWER} alebo
RNAView \citenum{RNAVIEW}.
Avšak iba niektoré z týchto nástrojov a algoritmov sú použiteľné pre vizualizáciu
veľkých štruktúr, akou sú napríklad podjednotky rRNA (RNAplot, RnaViz a RNAView).

Vzhľadom k tomu, že je nekonečne mnoho možností ako rozložiť sekundárnu štruktúru,
potrebujeme zistiť aké kritéria by malo nakreslenie RNA splňovať. Nanešťastie
tieto kritéria niesú formalizované, avšak niektoré vlastnosti ako napríklad
rovinnosť nakreslenia, kreslenie loopov (hairpin, bulge, interior a multi-branch)
na kružnice a vrcholy stemu ležiace na jednej priamke sú spoločné pre väčšinu
vizualizácii používaných vo vedeckej komunite \citenum{RNA_DRAW}.
Ostatné sa prispôsobujú oblasti štúdia, kvôli čomu každý algoritmus nebude
vyhovovať veľkej časti používateľov.
Dôvod môžeme ilustrovať na ribozomálnej RNA. Štruktúry týchto molekúl
majú veľké konzervované časti, ktoré biológovia očakávajú na rovnakom mieste
na obrázku, vďaka čomu sa vedia orientovať aj v relatívne veľkej molekule
a môžu skúmať a nachádzať tie menej konzervované časti, ktoré sa líšia medzi organizmami.
To znamená, že ak chceme štruktúru vizualizovať, potrebujeme ukladať
konzervované časti vždy na rovnaké miesto v obrázku.

Potreba vizualizácie, ktorá by zachovávala čo najviac spomínaných vlastností
nás viedla k vytvoreniu nového vizualizačného algoritmu \citenum{ELIAS_HOKSZA} založeného
na použití šablónovej (vzorovej) molekuly. Algoritmus na vstupe vezme cieľovú
štruktúru, ktorú chceme vizualizovať a inú, podobnú štruktúru u ktorej poznáme jej
nakreslenie. Túto podobnú molekulu nazývame šablónou. Obe štruktúry sú
prevedené do ich stromovej reprezentácie. Následne nájdeme najkratšiu postupnosť
editačných operácií, ktoré prevedú strom molekuly šablónovej na vizualizovanú
molekulu a rovnako menia aj nakreslenie šablóny na nakreslenie cieľovej molekuly.
Vzhľadom k tomu, že editačné operácie ktoré menia nakreslenie odpovedajú
minimálnej editačnej vzdialeností medzi stromami, nakreslenie spoločných častí sa nemení.
Táto metóda je teda schopná vizualizovať sekundárnu štruktúru RNA
molekuly presne podľa zvyku biológov - podľa poskytnutého vzorového nakreslenia.


